\documentclass{article}

% Language setting
% Replace `english' with e.g. `spanish' to change the document language
\usepackage[english]{babel}

% Set page size and margins
% Replace `letterpaper' with `a4paper' for UK/EU standard size
\usepackage[a4paper,top=2cm,bottom=2cm,left=3cm,right=3cm,marginparwidth=1.75cm]{geometry}

% Useful packages
\usepackage{amsmath}
\usepackage{graphicx}
\usepackage[colorlinks=true, allcolors=blue]{hyperref}
\usepackage{xcolor}
\usepackage{listings}

\colorlet{mygray}{black!30}
\colorlet{mygreen}{green!60!blue}
\colorlet{mymauve}{red!60!blue}

\lstset{
  backgroundcolor=\color{gray!10},  
  basicstyle=\ttfamily,
  columns=fullflexible,
  breakatwhitespace=false,      
  breaklines=true,                
  captionpos=b,                    
  commentstyle=\color{mygreen}, 
  extendedchars=true,              
  frame=single,                   
  keepspaces=true,             
  keywordstyle=\color{blue},      
  language=c++,                 
  numbers=none,                
  numbersep=5pt,                   
  numberstyle=\tiny\color{blue}, 
  rulecolor=\color{mygray},        
  showspaces=false,               
  showtabs=false,                 
  stepnumber=5,                  
  stringstyle=\color{mymauve},    
  tabsize=3,                                     
  title=\lstname 
}


\lstnewenvironment{code}[2][]{%
  \lstset{%
    numbers = left,
    title   = #2,
    #1,
  }%
}{}

\title{Assignment 2.1}
\author{Steinarr Hrafn Höskuldsson}

\usepackage{fancyhdr}
\fancypagestyle{firststyle}
{
   \fancyhf{}
   \fancyhead[L]{Embedded Systems Programming}
   
   \renewcommand{\headrulewidth}{0pt} % removes horizontal header line
}
\begin{document}
\pagestyle{firststyle}
{\let\newpage\relax\maketitle}
\section*{Part 1}

I created timer\_msec.h and timer\_msec.cpp as can be seen in appendix \ref{appendix:code}.
\newline

The init() method sets the timer up to fire every second.
Running the main.cpp program given in the assignment results in the LED blinking at a frequency of 0.5Hz as expected. 
\newline

The init(int period\_ms) method sets the timer up to fire every period\_ms milliseconds. 
The maximum length of interval the timer supports is \(\lfloor 2^{16}*1024 / 16000\rfloor = 4194\) milliseconds.

\section*{Part 2}
I modified the example to include a duty cycle option. The code can be seen in Appendix \ref{appendix:code}.
\newline
with the duty cycle set to 20\% I reduced  the timer period until I stopped noticing the light blinking. I found it to be around 26 or 27 milliseonds. 28 ms was visibly flickering and 25 ms looked very solid.
\newpage
\section*{Part 3}
I added a method to the timer class called set that changes the duty cycle.
I the used the while loop present in the main function do vary the duty cycle. Running the program I observed the LED fading from very dim to very bright over the course of 5 seconds before abruptly turning off and repeating the fade in. 
\newpage
and this program accomplishes the task as outlined.

\appendix
\section{Code}\label{appendix:code}
\lstinputlisting[caption=timer\_msec.cpp]{EmbeddedSystemsAssignment2_1/src/timer_msec.cpp}

\lstinputlisting[caption=timer\_msec.h]{EmbeddedSystemsAssignment2_1/src/timer_msec.h}


% TODO: put these in columns next to each other?

\end{document}

