\documentclass{article}

% Language setting
% Replace `english' with e.g. `spanish' to change the document language
\usepackage[english]{babel}

% Set page size and margins
% Replace `letterpaper' with `a4paper' for UK/EU standard size
\usepackage[a4paper,top=2cm,bottom=2cm,left=3cm,right=3cm,marginparwidth=1.75cm]{geometry}

% Useful packages
\usepackage{amsmath}
\usepackage{graphicx}
\usepackage[colorlinks=true, allcolors=blue]{hyperref}
\usepackage{xcolor}
\usepackage{listings}

\colorlet{mygray}{black!30}
\colorlet{mygreen}{green!60!blue}
\colorlet{mymauve}{red!60!blue}

\lstset{
  backgroundcolor=\color{gray!10},  
  basicstyle=\ttfamily,
  columns=fullflexible,
  breakatwhitespace=false,      
  breaklines=true,                
  captionpos=b,                    
  commentstyle=\color{mygreen}, 
  extendedchars=true,              
  frame=single,                   
  keepspaces=true,             
  keywordstyle=\color{blue},      
  language=c++,                 
  numbers=none,                
  numbersep=5pt,                   
  numberstyle=\tiny\color{blue}, 
  rulecolor=\color{mygray},        
  showspaces=false,               
  showtabs=false,                 
  stepnumber=5,                  
  stringstyle=\color{mymauve},    
  tabsize=3,                                     
  title=\lstname 
}


\lstnewenvironment{code}[2][]{%
  \lstset{%
    numbers = left,
    title   = #2,
    #1,
  }%
}{}

\title{Assignment 1.2}
\author{Steinarr Hrafn Höskuldsson}

\usepackage{fancyhdr}
\fancypagestyle{firststyle}
{
   \fancyhf{}
   \fancyhead[L]{Embedded Systems Programming}
   
   \renewcommand{\headrulewidth}{0pt} % removes horizontal header line
}
\begin{document}
\pagestyle{firststyle}
{\let\newpage\relax\maketitle}
\section*{Part 1}
I created timer\_msc.h and timer\_msc.cpp as follows:

\lstinputlisting[caption=timer\_msec.cpp]{EmbeddedSystemsAssignment2_1/src/timer_msec.cpp}

\lstinputlisting[caption=timer\_msec.h]{EmbeddedSystemsAssignment2_1/src/timer_msec.h}

# TODO: put these in columns next to each other?

Running the main.cpp program given results in the LED blinking at a frequency of 0.5Hz as expected.

The maximum length of interval the timer wupports is r2
It uses no RAM at compile time and uses only 176 bytes of Flash memory.
\newpage
\section*{Part 2}
A digital out driver was constructed with these two files:


The program given like the program from Part 1 uses no RAM at compile time and this time only 162 bytes of Flash.
\newpage
\section*{Part 3}
A digital in driver was constructed with these two files:
\begin{lstlisting}{digital_in.h}
#include <stdint.h>

class Digital_in
{
public:
    Digital_in(int pin);
    void init();
    void init(bool pull_up);
    bool is_hi();
    bool is_lo();
    bool state();

private:
    uint8_t pinMask;
};

\end{lstlisting}


\begin{lstlisting}{digital_in.cpp}
#include "digital_in.h"
#include <avr/io.h>

Digital_in::Digital_in(int pin)
{
    pinMask = (1 << pin);
}

void Digital_in::init()
{
    // no pull up asked for
    init(false);
}

void Digital_in::init(bool pull_up)
{
    DDRB &= ~pinMask; // DDR bit to 0
    if (pull_up)
    {
        PORTB |= pinMask; // PORT bit to 1 for pull up
    }
    else
    {
        PORTB &= ~pinMask;
    }
    // set DDR bit to 0
}

bool Digital_in::is_hi()
{
    return (PINB & pinMask);
}

bool Digital_in::is_lo()
{
    return !is_hi();
}

\end{lstlisting}
\newpage
and this program accomplishes the task as outlined.
\begin{lstlisting}{main.cpp}
#include <util/delay.h>
#include <digital_out.h>
#include <digital_in.h>

int main()
{
  Digital_out led(5); // PB5 Arduino Nano built-in LED on D13

  led.init();

  Digital_in button(1); // PB1, pin 9 on Arduino Nano

  button.init(true); // use internal pull up

  // led.set_hi();
  while (true)
  {
    _delay_ms(150);
    // led.toggle();
    if (button.is_lo())
    {
      led.set_lo();
    }
    else
    {
      led.toggle();
    }
  }
}

\end{lstlisting}
\end{document}

