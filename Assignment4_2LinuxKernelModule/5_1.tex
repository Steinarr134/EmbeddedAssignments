\documentclass{article}

% Language setting
% Replace `english' with e.g. `spanish' to change the document language
\usepackage[english]{babel}

% Set page size and margins
% Replace `letterpaper' with `a4paper' for UK/EU standard size
\usepackage[a4paper,top=2cm,bottom=2cm,left=3cm,right=3cm,marginparwidth=1.75cm]{geometry}

% Useful packages
\usepackage{amsmath}
\usepackage{graphicx}
\usepackage[colorlinks=true, allcolors=blue]{hyperref}
\usepackage{xcolor}
\usepackage{listings}
\usepackage{multicol}
\colorlet{mygray}{black!30}
\colorlet{mygreen}{green!60!blue}
\colorlet{mymauve}{red!60!blue}



\lstdefinelanguage{makefile}{
otherkeywords={.SUFFIXES},
morekeywords={SUFFIX, CPP_,},
moredelim=[is][\color{mbleu}]{/*}{*/},
style=global,%
morecomment=[l][commentstyle]{\#},%
emphstyle={\color{vimvert}},%
moredelim=[s][\color{vimvert}]{\$(}{)}%
}

\lstdefinelanguage{cpp}{
  backgroundcolor=\color{gray!10},  
  basicstyle=\ttfamily,
  columns=fullflexible,
  breakatwhitespace=false,      
  breaklines=true,                
  captionpos=b,                    
  commentstyle=\color{mygreen}, 
  extendedchars=true,              
  frame=single,                   
  keepspaces=true,             
  keywordstyle=\color{blue},      
  language=c++,                 
  numbers=none,                
  numbersep=5pt,                   
  numberstyle=\tiny\color{blue}, 
  rulecolor=\color{mygray},        
  showspaces=false,
  showstringspaces=false,
  showtabs=false,                 
  stepnumber=5,                  
  stringstyle=\color{mymauve},    
  tabsize=3,                                     
  title=\lstname 
}
\lstset{language=cpp}
\lstnewenvironment{code}[2][]{%
  \lstset{%
    numbers = left,
    title   = #2,
    #1,
  }%
}{}

\title{Embedded Systems Programming \\ Assignment 4.2 \\ \large Linux Kernel Module}
\author{Steinarr Hrafn Höskuldsson}

\usepackage{fancyhdr}
\fancypagestyle{firststyle}
{
   \fancyhf{}
   \fancyhead[L]{Embedded Systems Programming}
   
   \renewcommand{\headrulewidth}{0pt} % removes horizontal header line
}

\newcommand{\mycomment}[1]{}
\begin{document}
\pagestyle{firststyle}
{\let\newpage\relax\maketitle}

\mycomment{
\begin{figure}[h]
    \centering
    \includegraphics[width=0.75\textwidth]{LAB3/Basic1.png}
    \caption{"Switch test" Breadboard set up}
    \label{fig:Switch_test}
\end{figure}

\lstinputlisting[caption=main.cpp in Part 3]{Assignment3_1StateBehaviour/src/main_prt3.cpp}

}


\section*{Part 1}
The primary UART was configured using the \verb!raspi-config! interface. 
\\
The symbol rate was found via the \verb!stty! command to be 0.
\\
After running minicom and lowering the baud rate to 1200.  The \verb!stty! command reports the baud rate on \verb"/dev/ttyS0" to be 1200.
\\
The difference between 1200 and 115200 baud is not noticeable when typing but obvious when pasting a long sentence into minicom.
\\
Another way to detect the difference is by turning on local echo and typing at different speeds:
\begin{verbatim}
    
    ttyyppiinngg  sslloowwllyy
    
    ttyyppiningg  ffaastst..
\end{verbatim}

\\
After closing minicom the speed according to the \verb"stty" command goes back to 0.


\section*{Part 2}

Running the example given, the output was:
\begin{verbatim}
The following was read in [20]: Hello Raspberry Pi!
\end{verbatim}
Adding a newline character to the end of the string results in almost the same output only a second empty line is also printed. This is because in the \verb!printf! statement there is a newline character and the string received now also contains a newline character resulting in two consecutive newline characters.

The example was edited to read a line from stdin and send it to the serial port. For anything to be sent a newline has to be sent to stdin.

\\
After exiting the program the baud rate stays at 57600 according to the \verb"stty" command. Unlike when using \verb"minicom". The program does nothing to reset the settings on \verb"/dev/ttyS0".



\section*{Part 3}
Two c programs were written. \verb!part3_write! increments a counter once every second and writes its value to the serial port in binary format. \verb!part3_read! uses a blocking read to read a \verb!uint32_t! and prints it.

It took a bit of trial and error to get the read to always wait for 4 bytes. Before that was successfull the output looked like :
\begin{verbatim}
1
0
0
2
0
0
0
3
0
0
0
...
\end{verbatim}
 
 However after getting the port settings right the read call blocked until 4 bytes were read and then the output was as expected:
 \begin{verbatim}
1
2
3
...
.
...
255
256
257
...
 \end{verbatim}

\section*{Appendix}
\appendix
\subsection*{Part 2}\label{appendix:part2}
%\lstinputlisting[caption=include/increment.h]{Assignment4_1EmbeddedLinux/include/increment.h}
\subsection*{Part 3}\label{appendix:part3}
%\lstinputlisting[caption=include/decrement.h]{Assignment4_1EmbeddedLinux/include/decrement.h}
%\lstinputlisting[caption=src/increment.cpp]{Assignment4_1EmbeddedLinux/src/increment.cpp}

\end{document}

