\documentclass{article}

% Language setting
% Replace `english' with e.g. `spanish' to change the document language
\usepackage[english]{babel}

% Set page size and margins
% Replace `letterpaper' with `a4paper' for UK/EU standard size
\usepackage[a4paper,top=2cm,bottom=2cm,left=3cm,right=3cm,marginparwidth=1.75cm]{geometry}

% Useful packages
\usepackage{amsmath}
\usepackage{graphicx}
\usepackage[colorlinks=true, allcolors=blue]{hyperref}
\usepackage{xcolor}
\usepackage{listings}
\usepackage{multicol}
\colorlet{mygray}{black!30}
\colorlet{mygreen}{green!60!blue}
\colorlet{mymauve}{red!60!blue}



\lstdefinelanguage{makefile}{
otherkeywords={.SUFFIXES},
morekeywords={SUFFIX, CPP_,},
moredelim=[is][\color{mbleu}]{/*}{*/},
style=global,%
morecomment=[l][commentstyle]{\#},%
emphstyle={\color{vimvert}},%
moredelim=[s][\color{vimvert}]{\$(}{)}%
}

\lstdefinelanguage{cpp}{
  backgroundcolor=\color{gray!10},  
  basicstyle=\ttfamily,
  columns=fullflexible,
  breakatwhitespace=false,      
  breaklines=true,                
  captionpos=b,                    
  commentstyle=\color{mygreen}, 
  extendedchars=true,              
  frame=single,                   
  keepspaces=true,             
  keywordstyle=\color{blue},      
  language=c++,                 
  numbers=none,                
  numbersep=5pt,                   
  numberstyle=\tiny\color{blue}, 
  rulecolor=\color{mygray},        
  showspaces=false,
  showstringspaces=false,
  showtabs=false,                 
  stepnumber=5,                  
  stringstyle=\color{mymauve},    
  tabsize=3,                                     
  title=\lstname 
}
\lstset{language=cpp}
\lstnewenvironment{code}[2][]{%
  \lstset{%
    numbers = left,
    title   = #2,
    #1,
  }%
}{}

\title{Embedded Systems Programming \\ Assignment 4.1 \\ \large Embedded Linux}
\author{Steinarr Hrafn Höskuldsson}

\usepackage{fancyhdr}
\fancypagestyle{firststyle}
{
   \fancyhf{}
   \fancyhead[L]{Embedded Systems Programming}
   
   \renewcommand{\headrulewidth}{0pt} % removes horizontal header line
}

\newcommand{\mycomment}[1]{}
\begin{document}
\pagestyle{firststyle}
{\let\newpage\relax\maketitle}

\mycomment{
\begin{figure}[h]
    \centering
    \includegraphics[width=0.75\textwidth]{LAB3/Basic1.png}
    \caption{"Switch test" Breadboard set up}
    \label{fig:Switch_test}
\end{figure}

\lstinputlisting[caption=main.cpp in Part 3]{Assignment3_1StateBehaviour/src/main_prt3.cpp}

}

\section*{Part 1}

Using Rpi-Imager, a SD card was formatted and the Raspberry Pi was then booted. A terminal window was used to confirm that one could ssh onto it. SSH FS extension for VSCode was installed and after configuring SSH FS, the SSH client could connect to the Raspberry Pi. A workspace was created and a simple program was written that prints "Hello" to stdout.

\begin{lstlisting}[caption={src/hello.cpp, writes "Hello" to stdout}]
#include <stdio.h>

int main()
{
    printf("Hello\n");
    return 0;
}
\end{lstlisting}

It compiled and ran as expected. Next a makefile was created:

\begin{lstlisting}[language=makefile, caption={Makefile for simple hello program}]
# This is the default target, which will be built when 
# you invoke make
.PHONY: all
all: hello
# This rule tells make how to build the file hello from hello.cpp
hello: src/hello.cpp
	g++ -o hello src/hello.cpp
# This rule tells make to copy hello to the binaries subdirectory,
# creating it if necessary
.PHONY: install
install:
	mkdir -p bin
	cp -p hello bin
# This rule tells make to delete hello 
.PHONY: clean 
clean:
	rm -f hello

\end{lstlisting}
\section*{Part 2}
The Mutex example was copied and split into header files in \verb!include!, implementation files in \verb!src! and \verb!src/main_count.cpp!. They can be seen in Appendix \ref{appendix:crements}. A Makefile was created and configured such that it will only compile the changed implementation files.

\begin{lstlisting}[language=makefile, caption={Makefile for compiling Mutex example}]
BUILD=build
# This is the default target, which will be built when 
# you invoke make
.PHONY: all
all: main_count
# This rule tells make how to build the file hello from hello.cpp

hello: src/hello.cpp
	g++ -o hello src/hello.cpp

# build source files into objects files
$(BUILD)/%.o : src/%.cpp
	g++ -pthread -o $@ -c $< -I include
	
main_count: $(BUILD)/main_count.o $(BUILD)/decrement.o $(BUILD)/increment.o
	g++ -pthread -o $@ $^
#	g++ -pthread -o main src/main.cpp -I include

# This rule tells make to copy hello to the binaries subdirectory,
# creating it if necessary
.PHONY: install
install:
	mkdir -p bin
	cp -p hello bin
# This rule tells make to delete hello 
.PHONY: clean 
clean:
	rm -f hello
	rm -f $(BUILD)/*.o
	rm -f main

\end{lstlisting}

\section*{Part 3}
The \verb!increment! and \verb!decrement! implementations were copied to \verb!increment_fifo! and \verb!decrement_fifo! and modified so that increment adds an integer to the queue and \verb!decrement! gets an integer, the implementations can be seen in Appendix \ref{appendix:fifoments}. The \verb!Fifo! implementation was modified so that the inner workings of it were protected with the MutEx.
%\lstinputlisting[caption=Thread safe implementation of a First-in-First-out Queue. ]{Assignment4_1EmbeddedLinux/src/fifo3.cpp}
\verb!main_fifo.cpp! was written to start the \verb!increment! and \verb!decrement! threads. 

%\lstinputlisting[caption=src/main\_fifo.cpp]{Assignment4_1EmbeddedLinux/src/main_fifo.cpp}

\section*{Appendix}
\appendix
\section{Mutex example, split into header and implementation files}\label{appendix:crements}
\lstinputlisting[caption=include/increment.h]{Assignment4_1EmbeddedLinux/include/increment.h}
\lstinputlisting[caption=include/decrement.h]{Assignment4_1EmbeddedLinux/include/decrement.h}
\lstinputlisting[caption=src/increment.cpp]{Assignment4_1EmbeddedLinux/src/increment.cpp}
\lstinputlisting[caption=src/decrement.cpp]{Assignment4_1EmbeddedLinux/src/decrement.cpp}
\lstinputlisting[caption=src/main\_count.cpp]{Assignment4_1EmbeddedLinux/src/main_count.cpp}
\section{Mutex Fifo Implementation}\label{appendix:fifoments}
%\lstinputlisting[caption=include/increment\_fifo.h]{Assignment4_1EmbeddedLinux/include/increment_fifo.h}
%\lstinputlisting[caption=include/decrement\_fifo.h]{Assignment4_1EmbeddedLinux/include/decrement_fifo.h}
%\lstinputlisting[caption=src/increment\_fifo.cpp]{Assignment4_1EmbeddedLinux/src/increment_fifo.cpp}
%\lstinputlisting[caption=src/decrement\_fifo.cpp]{Assignment4_1EmbeddedLinux/src/decrement_fifo.cpp}

\end{document}

